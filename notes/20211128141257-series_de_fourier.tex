% Created 2021-11-29 seg 17:14
% Intended LaTeX compiler: pdflatex
\documentclass[11pt]{article}
\usepackage[utf8]{inputenc}
\usepackage[T1]{fontenc}
\usepackage{graphicx}
\usepackage{grffile}
\usepackage{longtable}
\usepackage{wrapfig}
\usepackage{rotating}
\usepackage[normalem]{ulem}
\usepackage{amsmath}
\usepackage{textcomp}
\usepackage{amssymb}
\usepackage{capt-of}
\usepackage{hyperref}
\usepackage{blindtext}
\usepackage{subcaption}
\usepackage{titling}
\usepackage{titlesec}
\usepackage{amsmath, amsfonts, amssymb}
\usepackage{multicol}
\usepackage{gensymb}
\usepackage{cancel}
\usepackage{siunitx}
\usepackage{colortbl,hhline}
\usepackage[brazil, english]{babel}
\usepackage[numbers]{natbib}
\usepackage{indentfirst}
\usepackage{graphicx}
\usepackage{siunitx}
\usepackage{chemfig}
\usepackage[backend=bibtex,bibencoding=ascii,style=chem-angew,citestyle=numeric-comp,sorting=none]{biblatex}
\usepackage{tikz}
\usetikzlibrary{arrows}
\usepackage[controls]{animate}
\usepackage{color}
\usepackage{epigraph}
\usepackage{empheq}
\usepackage{indentfirst}
\usepackage[theorems,skins]{tcolorbox}
\usepackage[svgnames]{xcolor}
\usepackage[explicit]{titlesec}
\usepackage[backend=bibtex,style=verbose-trad2]{biblatex}
\usepackage{physics}
\usepackage[svgnames]{xcolor}
\usepackage{framed}
\DeclareMathOperator{\sen}{sen}
\usepackage[margin=0.8in]{geometry}
\titleformat{\section}{\Large\bfseries}{\thesection}{2.5em}{}
\titlespacing{\section}{0em}{1.5em}{1.5em}
\titleformat{\subsection}{\large\bfseries}{\thesection}{2.5em}{}[\hrule]
\titlespacing{\section}{0em}{1.5em}{1.5em}
\author{Edgard Macena Cabral}
\date{\today}
\title{Séries de Fourier}
\hypersetup{
 pdfauthor={Edgard Macena Cabral},
 pdftitle={Séries de Fourier},
 pdfkeywords={},
 pdfsubject={},
 pdfcreator={Emacs 27.2 (Org mode 9.4.4)}, 
 pdflang={English}}
\begin{document}

\maketitle
\tableofcontents


\section{Introdução física pela equação de calor}
\label{sec:orgfe21547}

Fourier descobriu as séries de Fourier estudando soluções da
\href{20211128141648-termodinamica.org}{equação do calor} em uma dimensão (numa barra, por exemplo):
\begin{equation}
\label{eq:org40a9b74}
\pdv{u}{t} = k \pdv[2]{u}{x} 
\end{equation}

(Em que por alguma razão fizeram \(u\) ser a temperatura. Lindo!)

Como é comum em problemas de equações diferenciais, precisamos de mais
restrições para termos um problema realmente interessante. Podemos supor
então que ambas as extremidades da barra estão em equilíbrio termico com
blocos de gelo colocados ao lado, o que podemos escrever como:

\begin{equation}
\label{eq:orgc5f7ad4}
\begin{cases}
  u(x,t) &\ne 0 \\
  u_t(x.t) &= ku_{xx}(x,t)\\
  u(0,t)&=u(L,t) = 0 \mbox{ ,  } t > 0
  \end{cases}
\end{equation} 

Como consideração final, vamos supor que \(u(x,t)\) possa ser escrito como \(u =
  F(x)G(t)\). Assim, podemos escrever (\ref{eq:orgc5f7ad4}) como

 \begin{align*}
 u_{t} = \dot{G}F &= kGF'' = ku_{x x} \quad  \forall x,t \in \mathbb{R}\\
 &\Rightarrow \frac{1}{k} \frac{G'}{G} = \frac{F''}{F} = \lambda\\
  F(0)G(t) &= F(L)G(t) = 0\\
&\Rightarrow F(0) = F(L) = 0
 \end{align*}

Se você pensar um pouco, a segunda implica que \(\frac{G'}{G}\)
e \(\frac{F''}{F}\) são constantes, já que elas dependem de variáveis
diferentes. Daí a última igualdade com \(\lambda\). Além disso, se tívessemos \(G(t) =
 0\), \(u(x,t) = 0\), o que é contra nossas condições (\ref{eq:orgc5f7ad4}). Por fim, note que a segunda linha nos dá duas equações
diferenciais, e última nos dá duas condições de contorno!!

Maluco namoral quanta coisa numa brincadeira dessas. Pelo menos a primeira
equação diferencial é fácil de resolver
Para ela, em termos de \(G\), temos
\begin{align*}
\frac{\dot{G}(t)}{G(t)} &= k\lambda\\
\ln'(G(t)) & = k\lambda \\
\ln(G(t)) &= k\lambda t \Rightarrow (t)G = e^{k\lambda t}   
\end{align*}

Enfimm, enfim. Vamos resolvendo a segunda equação diferencial analizando caso a caso o
valor de \(\lambda\) como maior, igual, ou menor que zero.

\subsection{Supondo \(\lambda\) > 0:}
\label{sec:orgb41e836}

Para a segunda equação:
 \begin{gather*}
  \frac{F''}{F} = \lambda\\
  F'' - \lambda F = 0\\
  F = \alpha \sinh(x\sqrt{\lambda}) + \beta \cosh(x\sqrt{\lambda} )
\end{gather*}
Para a segunda equação, temos ainda que impor as restrições de \(F(0, t)\) e
\(F(L,t)\). A primeira implica em \(\beta = 0\), a segunda, em \(\alpha = 0\). Mas isso
nos dá \(u = 0\), então não temos a solução que exigimos.

\subsection{Supondo \(\lambda\) = 0.}
\label{sec:org873dae9}

Nesse caso
\[
   F'' = 0 \Rightarrow (F')' = 0 \Rightarrow F' = \alpha \Rightarrow F = \alpha x + \beta
   \]

Novamente, as condições de contorno de \(F\) implicam em \(\beta\) = 0 e \(\alpha\) = 0
respectivamente. Not good

\subsection{Supondo \(\lambda\) < 0}
\label{sec:org04627f0}
Nesse caso, vamos tratar de |\(\lambda\)|, okay? Ói:
 \begin{gather*}
  \frac{F''}{F} = -|\lambda|\\
F'' +  |\lambda| F = 0\\
F = \alpha \sen\qty(x\sqrt{|\lambda|}) + \beta \cos(x\sqrt{|\lambda|} )
 \end{gather*}

O QUE É LINDO, ESSA DE FATO PODE SER RESOLVIDA PELAS CONDIÇÕES DE CONTORNO
AAAA !!!1!!11!!
\end{document}