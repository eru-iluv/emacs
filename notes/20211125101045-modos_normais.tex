% Created 2021-11-25 qui 12:08
% Intended LaTeX compiler: pdflatex
\documentclass[11pt]{article}
\usepackage[utf8]{inputenc}
\usepackage[T1]{fontenc}
\usepackage{graphicx}
\usepackage{grffile}
\usepackage{longtable}
\usepackage{wrapfig}
\usepackage{rotating}
\usepackage[normalem]{ulem}
\usepackage{amsmath}
\usepackage{textcomp}
\usepackage{amssymb}
\usepackage{capt-of}
\usepackage{hyperref}
\usepackage{blindtext}
\usepackage{subcaption}
\usepackage{titling}
\usepackage{titlesec}
\usepackage{amsmath, amsfonts, amssymb}
\usepackage{multicol}
\usepackage{gensymb}
\usepackage{cancel}
\usepackage{siunitx}
\usepackage{colortbl,hhline}
\usepackage[brazil, english]{babel}
\usepackage[numbers]{natbib}
\usepackage{indentfirst}
\usepackage{graphicx}
\usepackage{siunitx}
\usepackage{chemfig}
\usepackage[backend=bibtex,bibencoding=ascii,style=chem-angew,citestyle=numeric-comp,sorting=none]{biblatex}
\usepackage{tikz}
\usetikzlibrary{arrows}
\usepackage[controls]{animate}
\usepackage{color}
\usepackage{epigraph}
\usepackage{empheq}
\usepackage{indentfirst}
\usepackage[theorems,skins]{tcolorbox}
\usepackage[svgnames]{xcolor}
\usepackage[explicit]{titlesec}
\usepackage[backend=bibtex,style=verbose-trad2]{biblatex}
\usepackage{physics}
\usepackage[svgnames]{xcolor}
\usepackage{framed}
\DeclareMathOperator{\sen}{sen}
\usepackage[margin=0.8in]{geometry}
\titleformat{\section}{\Large\bfseries}{\thesection}{2.5em}{}
\titlespacing{\section}{0em}{1.5em}{1.5em}
\titleformat{\subsection}{\large\bfseries}{\thesection}{2.5em}{}[\hrule]
\titlespacing{\section}{0em}{1.5em}{1.5em}
\author{Edgard Macena Cabral}
\date{\today}
\title{Modos normais}
\hypersetup{
 pdfauthor={Edgard Macena Cabral},
 pdftitle={Modos normais},
 pdfkeywords={},
 pdfsubject={},
 pdfcreator={Emacs 27.2 (Org mode 9.4.4)}, 
 pdflang={English}}
\begin{document}

\maketitle
\tableofcontents



\section{Uma introdução}
\label{sec:org9d342e2}

Antes de estabelecer a base do problema geral de oscilações acopladas,
façamos a análise de um exemplo simples.

Se \(x_{1}\) e \(x_{2}\) forem o deslocamentos do sistemam, sabemos que
\[
T  = \frac{1}{2}M(\dot{x}_{1}^{2} + \dot{x}_{2}^{2})
\]
e
\[
V = \frac{kx_{1}^{2}}{2} + \frac{kx_{2}^{2}}{2} + \frac{\kappa (x_{1} - x_{2})^2}{2} 
\]

Pela equações \href{20211123091021-lagrangiana.org}{Lagrangiana}, temos
\begin{align}
  M\ddot{x}_{1} \\
  M\ddot{x}_{2} + kx_{2} -\kappa (x_{1} - x_{2}) 
\end{align}

Façamos \(x_{j} = \Re(z_{j}) = B_{j}e^{i \omega t}\)
Temos:
\[\begin{cases}
(k + \kappa - M \omega^{2})B_{1} - \kappa B_{2} &= 0\\
-\kappa B_{1} + (k + \kappa - M \omega^{2})B_{2} &= 0
\end{cases}\]
Soluções não triviais implicam no determinante desse sistema ser diferente
de 0. Para isso, temos
\begin{align}
  (k + \kappa - M\omega^{2})^{2} -&\kappa^{2} = 0 \Rightarrow \omega^{2} = \frac{k + \kappa \pm \kappa }{M}\\
  &\omega_{a}^{2} = \frac{k + 2\kappa}{M} \mbox{ e } \omega_{b}^{2} = \frac{k}{M}
\end{align}

Para \(\omega = \omega_b \Rightarrow B_{1} = B_{2}\)
Já para \(\omega = \omega_a \Rightarrow B_1 = -B_2\)

$$\boxed{ \mbox{Nota: A nossas icógnitas aqui são } B_{a}^{\pm} \mbox{ e } B_{b}^{\pm} }$$

E assim resolvemos nosso problema em termo de \(x_{1} \mbox{ e } x_{2}\) e suas
derivadas.
 
\[\begin{cases}
  M\ddot{z}_{1} + (k + \kappa)z_{1} - \kappa z_{2} &= 0\\
  M\ddot{z}_{2} + (k + \kappa)z_{2} - \kappa z_{1} &= 0
\end{cases}\]
 
Ao somar e subtrair ambas as equações

\(M(\ddot{z}_{1} \pm \ddot{z}_{2}) + (k + \kappa)(\ddot{z}_{1} \pm \ddot{z}_{2}) - \kappa(z_{2}
\pm z_{1})\)

Fazendo

\begin{align}
\gamma_{a} &\equiv z_{1} - z_{2} \\
\gamma_{b} &\equiv z_{1} + z_{2}
\end{align}

Assim:
\[\begin{cases}
 M\ddot{\gamma_{a}} + (k + 2\kappa)\gamma_{a} = 0\\
 M\ddot{\gamma}_{b} + k\gamma_{b} = 0 
\end{cases}\]
O que nos dá as equações desacopladas cujas equações são independentes

\begin{align}
  \gamma_{a} = C_{1}^{+ } e^{i\omega_{a}t} + C_{1}^{-}e^{-i\omega_{a}t}\\ 
  \gamma_{b} = C_{2}^{+} e^{i\omega_{a}t} + C_{2}^{-}e^{-i\omega_{a}t}
\end{align}

Embora isso seja \emph{bemmmm} mais simples, ela tem a desvantagem de não ser a
solução dos corpos separados. Perceba que \(\gamma_{a} = 0\) para qualquer \(t\), \(x_{1}
= x_{2}\) e as distâncias entre as massas permanece a mesma, é simétrica. Já para \(\gamma_{b} = 0\), \(x_{1} = -x_{2}\) e o
movimento é anti-simétrico.

\section{Generalizando o problema}
\label{sec:org03829cc}
\end{document}