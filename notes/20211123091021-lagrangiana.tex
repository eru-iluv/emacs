% Created 2021-11-23 ter 17:38
% Intended LaTeX compiler: pdflatex
\documentclass[11pt]{article}
\usepackage[utf8]{inputenc}
\usepackage[T1]{fontenc}
\usepackage{graphicx}
\usepackage{grffile}
\usepackage{longtable}
\usepackage{wrapfig}
\usepackage{rotating}
\usepackage[normalem]{ulem}
\usepackage{amsmath}
\usepackage{textcomp}
\usepackage{amssymb}
\usepackage{capt-of}
\usepackage{hyperref}
\usepackage{tikz}
\usepackage{blindtext}
\usepackage{subcaption}
\usepackage{titling}
\usepackage{titlesec}
\usepackage{amsmath, amsfonts, amssymb}
\usepackage{multicol}
\usepackage{gensymb}
\usepackage{cancel}
\usepackage{siunitx}
\usepackage{colortbl,hhline}
\usepackage[brazil, english]{babel}
\usepackage[numbers]{natbib}
\usepackage{indentfirst}
\usepackage{graphicx}
\usepackage{siunitx}
\usepackage{chemfig}
\usepackage[backend=bibtex,bibencoding=ascii,style=chem-angew,citestyle=numeric-comp,sorting=none]{biblatex}
\usepackage{tikz}
\usetikzlibrary{arrows}
\usepackage[controls]{animate}
\usepackage{color}
\usepackage{epigraph}
\usepackage{empheq}
\usepackage{indentfirst}
\usepackage[theorems,skins]{tcolorbox}
\usepackage[svgnames]{xcolor}
\usepackage[explicit]{titlesec}
\usepackage[backend=bibtex,style=verbose-trad2]{biblatex}
\usepackage{physics}
\usepackage[svgnames]{xcolor}
\usepackage{framed}
\DeclareMathOperator{\sen}{sen}
\usepackage[margin=0.8in]{geometry}
\titleformat{\section}{\Large\bfseries}{\thesection}{2.5em}{}
\titlespacing{\section}{0em}{1.5em}{1.5em}
\titleformat{\subsection}{\large\bfseries}{\thesection}{2.5em}{}[\hrule]
\titlespacing{\section}{0em}{1.5em}{1.5em}
\author{Edgard Macena Cabral}
\date{\today}
\title{Lagrangiana}
\hypersetup{
 pdfauthor={Edgard Macena Cabral},
 pdftitle={Lagrangiana},
 pdfkeywords={},
 pdfsubject={},
 pdfcreator={Emacs 27.2 (Org mode 9.4.4)}, 
 pdflang={English}}
\begin{document}

\maketitle
\tableofcontents




\section{Resumo}
\label{sec:orgab3b1c9}
Vimos nas aulas anteriores que a análise do princío de Hamilton fornecia as equações de movimento em qualquer coordenadas

Vimos ainda que para a equação

\begin{equation}
 L = T - V
\end{equation}

Obtemos, através de Euler-Lagrange
\begin{equation}\label{eq:Euler-Lagrange}
\label{eq:orgcc08765}
\frac{dL}{d\mathbf{r}} - \frac{d}{dt}\left(\dfrac{dL}{d\mathbf{\dot{r}}}\right) = 0
\end{equation}

Nos fornece as equações do momento e força generalizados (angular e linear)
Façamos um resumo sobre quando devemos usar multiplicadores de Lagrange no
formalismo Lagrangiano

O ponto de partida deve ser observar que o formalismo de lagrange está
baseado no princípio de Hamilton. Por sua vez, vimos que se não há vínculos
entre as coordenadas e o movimento, podemos aplicar de maneira simples Euler
Lagrange. Se houver vínculos

Portano, vemos que o uso de multiplicadores de Lagrange depende do funcional
L (Lagrangiano) ser função de variáveis que tem vínculos entre si.

Podemos ainda generalizar o que vimos até agora para \emph{N} párticulas

Tendo a revisão feita, vamos partir pra generalização para qualquer
coordenada.


\section{Coordenadas generalizadas e graus de liberdade}
\label{sec:org89ff88e}
Dizemos que \(q_1, q_2, ..., q_n\) é um conjunto de \emph{coordenadas generalizadas} se
a posição de toda partícula do sistema é uma função dessas variáveis

\begin{equation}\label{eq:coordenadas generalizadas}
\mathbf{r_i} = \mathbf{r_i}(q_1, q_2, ..., q_n)
\end{equation}

em que \(\mathbf{r_i}\) é a i-ésima coordenada

\begin{itemize}
\item Pendulo simples
\(\mathbf{r} = \mathbf{r}(x,y)\)
\(\mathbf{r} = \mathbf{r}(\theta)\)

São ambas coordenadas generalizadas de \(\mathbf{r}\)
\end{itemize}

O número de graus de liberdade de um sistema é dado pelo número de
coordenadas que podem variar independentemente

Se o número de coordenadas generalizadas é igual ao de graus de liberdade, o
conjunto é chamado de próprio
\subsection{Outras definições}
\label{sec:orgef04453}
\begin{itemize}
\item Sistema holonômico:
Sistema no qual as equações de vínculo podem ser usadas pra eliminar
algumas coordenadas, reduzindo-se o número total de coordenadas. Vinculos
do tipo \(g(\{q_i\},t = 0)\) são holonômicos
\item Sistema forçado
Se a eliminação de coordenadas introduzir funções explícitoas do tempo, o
sistema é dito forçado, se os vínculos são tais que que \(\mathbf{r}\)
não depende explicitamente de \(t\), o sistema é dito \textbf{natural}

Se o sistema for natural, \(T\) é a soma de das energias cinéticas devido as
energias cinéticas generalizadas, apenas com termos quadráticos.
\end{itemize}

Vamos agora determinar as equações de movimento em termos de coordenadas
generalizadas.

Se \(\{q_i\}\) for um conjunto próprio, e o sistema tiver \(s\) \uline{graus de liberdade}

  \begin{equation}
equação
  \end{equation}

Obtivemos tal resultado considerando:
\begin{enumerate}
\item As forças consideradas até agora são todas conservativas ou de vínculo
\item Os vínculos usados até aqui são todos holonômicos
\end{enumerate}

\section{Leis de conservação}
\label{sec:org891f687}
Observe que se uma coordenada própria não aparecer explicitamente na
lagrangiana, equação para essa coordenada sera

  \begin{equation}
\frac{\partial L}{\partial q_{\alpha}} - \frac{d}{dt}\left(\frac{\partial L}{\partial \dot{q}_{\alpha}} \right) = 0
  \end{equation}

Temos então
 \begin{equation}
\frac{d}{dt}\left(\frac{\partial L}{\partial \dot{q}_{\alpha}}\right) = 0
 \end{equation}
Que é uma constante do movimento

\section{Forças não inerciais}
\label{sec:orge392d5a}
Até o momento só trabalhamos com forças que fossem derivadas de funções
energias potenciais, mas podemos fazer uma análise para casos em que a força
generalizadas podem ser escritas como:
$$\boxed{F_{\alpha} = -\pdv{U}{q_{\alpha }} + \dv{t}(\pdv{U}{q_{\alpha}})}$$
Nesses casos, podemos definir \(L = T - U\), onde \(U\) é a "energia potencial
efetiva do sistema". Nesse caso, a Lagrangiana pode ser escrita da equação de
Lagrange.

Consideremos um sistema de 1 partícula sujeita a uma força resultante
\(\mathbf{F\), cuja posição possa ser escrita em termos do conjunto
própri de coordenadas generalizadas \((q_{1},q_{2}, q_{3})\).

$$I = \int^{t_{2}}_{t_{1}}T\mathrm{d}x = \int^{t_{2}}_{t_{1}}T(q_{\alpha}, \dot{q}_{\alpha}, t)\mathrm{d}x $$

Note que \(\delta T = \delta\qty[\frac{1}{2}(\dot{x}^{2}+\dot{y}^{2} + \dot{z}^{2})]\)

podemos integrar por partes

\begin{equation}
\delta I = m(\dot{x}\delta x + \dot{y}\delta y + \dot{z}\delta z)\eval^{t_{2}}_{t_{1}}- \int^{ t_{2} }_{ t_{1} }m(\ddot{x}\delta x + \ddot{y}\delta y + \ddot{z}\delta z)\mathrm{d}t 
\end{equation}

Note que o termo da derivada corresponde a integral do tabalho virtual (que é o
trabalho para levar umaa partícula de um ponto a outro em um tempo
instântaneo).recentlyused:/locations

$$\delta I = - \int^{ t_{2} }_{ t_{1} }\delta W \mathrm{d}t$$

Podemos escrever \(\delta W = \sum_{\alpha=1}^{3}F_{\alpha}\delta q_{\alpha}\) (Essa é a definição de
\(F_{\alpha}\)).

Sabemos ainda que \ref{eq:orgcc08765}

$$\boxed{\delta I = \sum^{3}_{\alpha = 1}\pdv{ L }{ q_{\alpha} } - \dv{t}\left(\pdv{ L }{ \dot{q}_{\alpha} }\right)}$$


Unindo os dois, temos pra cada termo \(\alpha\)
\end{document}